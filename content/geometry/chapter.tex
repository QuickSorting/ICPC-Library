\chapter{Geometry}

\section{Geometric primitives}
	\kactlimport{Point.h}
	\kactlimport{lineDistance.h}
	\kactlimport{SegmentDistance.h}
	\kactlimport{SegmentIntersection.h}
	\kactlimport{lineIntersection.h}
	\kactlimport{sideOf.h}
	\kactlimport{OnSegment.h}
	\kactlimport{linearTransformation.h}
	% \kactlimport{LineProjectionReflection.h}
	\kactlimport{Angle.h}

\section{Circles}
	\kactlimport{CircleIntersection.h}
	\kactlimport{CircleTangents.h}
	\kactlimport{CircleLine.h}
	\kactlimport{CirclePolygonIntersection.h}
	\kactlimport{circumcircle.h}
	\kactlimport{MinimumEnclosingCircle.h}

\section{Polygons}
	\kactlimport{InsidePolygon.h}
	\kactlimport{PolygonArea.h}
	\kactlimport{PolygonCenter.h}
	\kactlimport{PolygonCut.h}
	% \kactlimport{PolygonUnion.h}
	\kactlimport{MinkowskiSum.h}
	\kactlimport{ConvexHull.h}
	\kactlimport{HullDiameter.h}
	\kactlimport{PointInsideHull.h}
	\kactlimport{LineHullIntersection.h}
	\kactlimport{HalfPlaneIntersection.h}
	\kactlimport{PointInPolygonOrTangentToPolygon.h}

\section{Misc. Point Set Problems}
	\kactlimport{ClosestPair.h}
	\kactlimport{ManhattanMST.h}
	\kactlimport{kdTree.h}
	% \kactlimport{DelaunayTriangulation.h}
	\kactlimport{FastDelaunay.h}

\section{3D}
	\subsection{Dot product:}
	\[
		\vec{v} \cdot \vec{w} = \lVert \vec{v} \rVert \lVert \vec{w} \rVert \cos \theta
	\]
	\subsection{Cross product:}
	If $\vec{v}$ and $\vec{w}$
	are parallel, $\vec{v} \times \vec{w} =
	\vec{0}$ , and otherwise it is defined as
	\[
		\vec{v} \times \vec{w} = (\lVert \vec{v} \rVert \lVert \vec{w} \rVert \sin \theta) \vec{n}
	\]
	where $\vec{n}$ is a unit vector perpendicular to both $\vec{v}$
	and $\vec{w}$ chosen using the right-hand rule. Note that the norm of the 3D cross
	product is equal to the absolute value of the 2D cross product. \\
	The right-hand rule says this: if you take your right hand, align your
	thumb with $\vec{v}$ and your extended index $\vec{w}$, and fold your middle finger at
	a 90◦ angle, then it will point in the direction of $\vec{v} \times \vec{w}$. 

	\subsection{Mixed product and orientation}
	A very useful combination of dot product and cross product is the mixed
	product. We define the mixed product of three vectors $\vec{u}$ , $\vec{v}$ and $\vec{w}$ as
	\[
		(\vec{u} \times \vec{v}) \cdot \vec{w}
	\]
	Let P be the plane containing $\vec{u}$ and $\vec{v}$. We know that $\vec{n} = \vec{u} \times \vec{v}$ is perpendicular to P, and $\vec{n} \cdot \vec{w}$ will be positive if the angle between $\vec{n}$ and $\vec{w}$ is less than 90◦. This will happen if $\vec{w}$ points to the same side of P as $\vec{n}$ ,
	while when $\vec{n} \cdot \vec{w}$ is negative $\vec{w}$ will point to the opposite side. \\
	Note that this is similar to how the 2D cross product $\vec{v} \times \vec{w}$ tells us to
	which side of to line containing  $\vec{v}$, the vector $\vec{w}$ points. So, we similarly define
	an $orient()$ function based on it:
	\[
		orient(P, Q, R, S) = (\vec{PQ} \times \vec{PR}) \cdot \vec{PS}
	\]
	It is positive if $S$ is on the side of plane $PQR$ in the direction of $\vec{PQ} \times \vec{PR}$,
	negative if $S$ is on the other side, and zero if $S$ is on the plane. \\
	Earlier, we implicitly assumed that $P$, $Q$, $R$ were not collinear, but in
	general $orient(P, Q, R, S)$ is zero if and only if $P$, $Q$, $R$, $S$ are coplanar, so
	when any three points are collinear it is always zero. We can also say that
	it is nonzero if and only if lines $PQ$ and $RS$ are skew, that is, neither
	intersecting nor parallel.
	Finally, $\lVert orient(P, Q, R, S) \rVert$ is equal to six times the volume of tetrahedron $PQRS$.


	\kactlimport{PolyhedronVolume.h}
	\kactlimport{Point3D.h}
	\kactlimport{3dHull.h}
	\kactlimport{sphericalDistance.h}
